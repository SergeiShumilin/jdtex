\documentclass[a4paper,10pt]{extarticle}                     % a4 document with font 14pt

\usepackage[top=2cm,left=2cm,right=2cm,bottom=2cm]{geometry} % margins
\usepackage[utf8]{inputenc}                                  % for MikTex editor support
\usepackage[T2A]{fontenc}                                    % for russian encoding support
\usepackage[russian]{babel}                                  % russian encoding on
\usepackage{amsmath,amsthm,amscd,amsfonts,amssymb}           % special symbols, etc.
\usepackage{indentfirst}                                     % paragraph first word indentation
\usepackage{textcomp}                                        % text in formulas
\usepackage{graphicx}                                        % graphics on

\numberwithin{equation}{section}                             % section numbering for formulas
\numberwithin{figure}{section}                               % section numbering for pictures

\theoremstyle{plain}                                         % theorem style
\newtheorem{theorem}{Теорема}[section]                       % theorem
\newtheorem{lemma}{Лемма}[section]                           % lemma
\newtheorem{definition}{Определение}[section]                % definition

\begin{document}

\title{TODO: name of article}
\author{Алексей Рыбаков \\ rybakov.aax@gmail.com}
\date{Март, 2015}
\maketitle

\section{Основные соглашения и определения}

Точку в пространстве будем задавать ее радиус-вектором - вектором, направленным из начала координат в рассматриваемую точку ($\overline{P}$).

Прямую в трехмерном пространстве будем задавать \textit{опорной точкой} $\overline{P_0}$ и \textit{направляющим вектором} $\overline{V}$ ($L = (\overline{P_0}, \overline{V})$).
Геометрическое место точек прямой описывается следующим образом:

\begin{equation}
    L = \{ \overline{P(t)} = \overline{P_0} + t\overline{V} \ | \ t \in \mathbb{R} \}.
\end{equation}

Сферу в трехмерном пространстве будем задавать ее центром $\overline{C}$ и радиусом $R$ ($S = (\overline{C}, R)$).
Геометрическое местов точек поверхности сферы описывается следующим образом:

\begin{equation}
    S = \{ \overline{P} \ | \ {|\overline{P} - \overline{C}|}^2 = R^2 \}.
\end{equation}

\begin{definition}
Пучком сфер, опирающимся на отрезок, концами которого являются точки пространства $\overline{P_1}$ и $\overline{P_2}$, будем называть семейство сфер $S(\alpha) = (\overline{C(\alpha)}, R(\alpha))$, где центры и радиусы сфер задаются следующими соотношениями:
\begin{equation}
    \begin{cases}
        \overline{C(\alpha)} = \overline{P_1} + \alpha(\overline{P_2} - \overline{P_1}) = \overline{P_1} + \alpha \overline{V_{12}}, \\
        R(\alpha) = R_1 + \alpha(R_2 - R_1) = R_1 + \alpha R_{12},
    \end{cases}
\end{equation}
где $\alpha \in [0; 1]$.
Обозначать пучок сфер будем
\begin{equation}
    \mathcal{S}(S(\alpha \in [0; 1])]) = \mathcal{S}(S_1, S_2),
\end{equation}
где $S_1 = (\overline{P_1}, R_1), \ S_2 = (\overline{P_2}, R_2)$.
\end{definition}

\end{document}

